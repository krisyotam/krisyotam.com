%-----------------------------%
% Template 4c: Legal Document Template
%-----------------------------%
\documentclass[11pt]{article}

%- Packages -%
\usepackage[utf8]{inputenc}
\usepackage[T1]{fontenc}
\usepackage{lmodern}
\usepackage[a4paper,margin=1in]{geometry}
\usepackage{setspace}
\usepackage{titlesec}
\usepackage{fancyhdr}
\usepackage{hyperref}

%- Title formatting -%
\newcommand{\TitleLine}{\noindent\rule{\textwidth}{0.4pt}}
\titleformat*{\section}{\bfseries\normalsize}
\titleformat*{\subsection}{\bfseries\normalsize}

%- Header / Footer -%
\pagestyle{fancy}
\fancyhf{}
\renewcommand{\headrulewidth}{0pt}
\fancyfoot[L]{\itshape © 2025 Kris Yotam}
\fancyfoot[R]{\itshape Last Updated: May 23, 2025}
\fancyfoot[C]{\thepage}

%- Document -%
\begin{document}
\onehalfspacing

%- Title block -%
\vspace*{-1em}
\TitleLine
\vspace{0.5em}
\begin{center}
  {\Large\bfseries AI Transparency Statement}\\[0.75em]
  {\normalsize Kris Yotam}\\
  {\itshape Illinois, United States}
\end{center}
\vspace{0.5em}
\TitleLine

\vspace{1.5em}

%- Preamble & Keywords -%
\noindent\textbf{Preamble}\\
This AI Transparency Statement discloses how artificial intelligence (AI) technologies are used on krisyotam.com. We believe in being transparent about when and how AI is used in content creation, site functionality, and user interactions. This statement outlines which AI systems we employ, their purposes, and our human oversight processes.

\vspace{0.75em}
\noindent\textit{Keywords:} AI Disclosure; Content Generation; Machine Learning; Generative AI; Human Review

\vspace{1.5em}

%- Main sections -%
\section{AI Model Usage}
krisyotam.com uses several AI models and systems for various purposes across the site. The following outlines our primary AI implementations:

\subsection{Kelex (Custom Fine-Tuned Model)}
\begin{itemize}
  \item \textbf{Model Type:} Large Language Model, fine-tuned on the DeepSeek architecture
  \item \textbf{Primary Use:} Generation of creative prose and poetry
  \item \textbf{Implementation:} Used to draft certain blog posts, essays, and poetic content
  \item \textbf{Disclosure Method:} Content primarily generated by Kelex is marked with an "AI-assisted" or "AI-generated" label at the beginning or end of the content
\end{itemize}

\subsection{GPT Models}
\begin{itemize}
  \item \textbf{Model Type:} OpenAI's GPT series language models
  \item \textbf{Primary Use:} Research assistance, content summarization, and drafting
  \item \textbf{Implementation:} Used for research synthesis, initial drafting of technical content, and generating content outlines
  \item \textbf{Disclosure Method:} Content with significant GPT contributions is labeled as "AI-assisted" in the publication notes
\end{itemize}

\subsection{Gemini (Google AI)}
\begin{itemize}
  \item \textbf{Model Type:} Google's multimodal AI system
  \item \textbf{Primary Use:} Visual content analysis and multimodal tasks
  \item \textbf{Implementation:} Used for image descriptions, analyzing visual data, and assisting with graphical content
  \item \textbf{Disclosure Method:} Visual or multimodal content created with Gemini assistance is labeled accordingly
\end{itemize}

\subsection{Other AI Tools}
We may occasionally use additional AI tools for specific purposes:
\begin{itemize}
  \item Image generation models (for creating illustrations or graphical elements)
  \item Audio processing AI (for podcast transcription or audio enhancement)
  \item Recommendation systems (for suggesting related content to readers)
\end{itemize}
When such tools are used, appropriate disclosure is provided in context.

\section{AI Content Identification}
Content on krisyotam.com that has been substantially created or assisted by AI is identified through one or more of the following methods:
\begin{itemize}
  \item An explicit statement at the beginning or end of the content (e.g., "This article was generated with AI assistance")
  \item A visual indicator or icon denoting AI involvement
  \item Information in the metadata or publication notes
  \item A dedicated section on the content page explaining the role AI played in creation
\end{itemize}

We are committed to clearly distinguishing between human-authored and AI-generated or AI-assisted content to maintain transparency with our audience.

\section{Human Review and Editing Process}
While we utilize AI as a creative and productivity tool, all AI-generated content undergoes human review before publication:
\begin{itemize}
  \item \textbf{Editorial Oversight:} All AI outputs are reviewed, edited, and approved by human editors
  \item \textbf{Fact-Checking:} Information presented in AI-drafted content is verified for accuracy
  \item \textbf{Quality Control:} Content is assessed for relevance, coherence, and alignment with our editorial standards
  \item \textbf{Original Contributions:} Human editors may substantially modify, expand upon, or reframe AI-generated content
\end{itemize}

The level of human involvement varies by content type and is generally indicated in the content's disclosure statement.

\section{AI in User Interactions}
Some features on krisyotam.com may use AI to enhance user experience:

\subsection{Comment Moderation}
AI systems may assist in filtering spam and inappropriate comments. All automated moderation decisions can be reviewed by human moderators upon request.

\subsection{Search and Content Discovery}
AI algorithms may help power search functionality and content recommendations on the site, helping users discover relevant content based on their interests and browsing behavior.

\subsection{User Interface Personalization}
Some UI elements may adapt based on AI analysis of usage patterns to provide a more relevant experience.

For any AI-powered interactive features, we strive to make the AI's role clear through appropriate interface cues or explicit disclosures.

\section{AI Training Data and Privacy}
Our approaches to AI and user data prioritize privacy:
\begin{itemize}
  \item \textbf{Custom Model Training:} Our fine-tuned model (Kelex) was trained primarily on public domain texts, classic literature, and original writings by the site owner. It does not use user data or comments for training.
  \item \textbf{Third-Party AI Services:} When we use external AI services (like OpenAI's GPT or Google's Gemini), we follow data minimization principles and avoid sending sensitive or personal information to these services.
  \item \textbf{User Content:} We do not use user-submitted content (such as comments) to train AI models without explicit consent.
\end{itemize}

For more information about data handling and privacy, please see our Privacy Policy.

\section{Quality and Limitations}
We acknowledge that AI-generated content has inherent limitations:
\begin{itemize}
  \item AI may occasionally produce inaccurate or misleading information
  \item AI may reproduce biases present in training data
  \item AI lacks genuine understanding of context and cannot truly verify facts
\end{itemize}

We implement editorial processes to mitigate these limitations, but we encourage readers to approach AI-generated content with appropriate critical thinking. If you identify potential errors or issues in our content, please contact us.

\section{Feedback and Concerns}
We value your feedback on our use of AI technologies. If you have questions, concerns, or suggestions regarding our AI practices or specific content, please contact us at:

\begin{quote}
  Kris Yotam\\
  \href{mailto:ai@krisyotam.com}{ai@krisyotam.com}
\end{quote}

\section{Regulatory Compliance}
Our AI transparency practices are designed to align with emerging AI regulations and ethical guidelines, including:
\begin{itemize}
  \item EU AI Act transparency requirements
  \item FTC guidance on truth in AI claims
  \item Industry best practices for AI disclosure
\end{itemize}

As regulatory frameworks evolve, we will update our practices to remain compliant with applicable laws and standards.

\section{Updates to This Statement}
This AI Transparency Statement will be updated as our use of AI technologies evolves. Please check back periodically for the most current information about how we use AI on krisyotam.com.

For more detailed information about our AI use cases and implementation, you can also visit our AI information page at \href{https://krisyotam.com/blog/2025/ai-use-cases}{krisyotam.com/blog/2025/ai-use-cases}.

\end{document}
